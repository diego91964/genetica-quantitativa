\chapter{Parâmetros Genéticos}

<<<<<<< HEAD
=======
\section{Herdabilidade}

A herdabilidade é um coeficiente que expressa a relação entre a variância genotípica e a variância fenotípica, ou seja, mede o nível da correspondência entre o fenótipo e o valor genético. 

\begin{definition}[Dedução de Herdabilidade]

\begin{align}
&  r = \frac{Cov(x,y)}{\sqrt{v(x) \times v(y)}} \\
&  r(F,G) = \frac{Cov(F,G)}{\sqrt{v(F) \times v(G)}} \\
&  r(F,G) = \frac{Cov(G+A,G)}{\sqrt{v(F) \times v(G)}} \\
&  r(F,G) = \frac{Cov(G,G) + Cov(G,A)}{\sqrt{v(F) \times v(G)}} \\
&  Como: Cov(G,A) = 0 \\
&  \rightarrow r(F,G) = \frac{Cov(G,G)}{\sqrt{v(F) \times v(G)}} \\
&  Como: Cov(X,X) = V(X) \\
&  \rightarrow r(F,G) = \frac{V(G)}{\sqrt{v(F) \times v(G)}} \\
&  r(F,G) = \sqrt{\frac{[V(G)]^2}{v(F) \times v(G)}} \\
&  r(F,G) = \sqrt{\frac{[V(G)]}{v(F)}} \\
&  Como: H = \frac{[V(G)]}{v(F)} \\
&  \rightarrow  r(F,G) = \sqrt{H^2} \\
\end{align}
\end{definition}


\begin{definition}[Fórmula da Herdabilidade]

\begin{align}
&  H^2 = \frac{V(G)}{V(F)} \\
\end{align}
\end{definition}

O valor da herdabilidade pode variar entre 0 e 1. Por definição, quando o valor da herdabilidade é maior que 0,7 é considerado alto para plantas. Em caso de animas, pode variar entre 0,3 e 0,4.



\subsection{Ganho de Seleção}

\begin{definition}[Fórmula do Ganho de Seleção (GS)]

\begin{align}
& GS = H^2 \times (\overline{X}_s - \overline{X}_0) \\
\end{align}
Sendo Xs a média dos indivíduos selecionados e X0 a média inicial dos indivíduos.
\end{definition}


\subsection{Média Predita}

\begin{definition}[Fórmula Média Predita (Xm)]

\begin{align}
& Xm = GS + \overline{X}_0 \\
\end{align}
\end{definition}


\subsection{Número de Genes}


\begin{definition}[Fórmula Número de Genes (Nrg)]

\begin{align}
& Nrg = \frac{(\overline{P_1} - \overline{P_2})^2}{8 \times V(G)_{F1}} \\
\end{align}
Sendo P1 a média dos parentais 1 e  P2 a média dos parentais 2 
\end{definition}

\subsection{Exemplos}

\textbf{Exemplo 1}

Consideral os seguintes dados :

\begin{table}[h]
\centering
\begin{tabular}{l l l l l}
\midrule
13,13 &  20,35 & 21,60 & 17,77 & 15,74 \\
19,90 &  24,92 & 16,81 & 22,91 & 15,68 \\
19,05 &  19,30 & 20,78 & 20,64 & 15,46 \\
17,14 &  19,21 & 15,45 & 17,88 & 18,43 \\
\end{tabular}
\caption{Dados Pai 1} \label{tab:t06}
\end{table}


\begin{table}[h]
\centering
\begin{tabular}{l l l l l}
\midrule
103,39 &  100,94 &  99,26  &  104,13 &  97,54  \\
96,24  &  106,47 &  105,45 &  95,40  &  104,52 \\
97,60  &  95,36  &  98,85  &  98,89  &  98,13  \\
98,56  &  100,37 &  105,99 &  104,95 &  98,85  \\
\caption{Dados Pai 2} \label{tab:t07}
\end{table}


\begin{example}[Example name]
\lipsum[2]
\end{example}
>>>>>>> d9ae7c9f946ee2728f07472aaf0c5ec90f2e60a3
