\chapter{Introdução à Genética de Populações}

Para este capítulo, admitir os seguintes termos:

\begin{enumerate}
\item 1 gene e dois alelos ('A' e 'a')
\item D: Frequência de Homozigotos A
\item H: Frequência de Heterozigoto
\item R: Frequência de Homozigotos a
\end{enumerate}


\begin{table}[H]
\centering
\begin{tabular}{l l l}
\toprule
 \textbf{Genótipos Possíveis} & \textbf{Frequência (f)} & \textbf{Frequência}  \\
\midrule
 AA &  n1/n & D \\
 Aa &  n2/n & H \\
 aa &  n3/n & R \\
\bottomrule
\end{tabular}
\end{table}


\section{Frequência Genica}


\begin{definition}[Frequência do alelo A]
\begin{align}
& Freq(A) = \frac{Nº alelos A}{Nº total de alelos}
& Freq(A) = \frac{n1}{n} + \frac{n1}{2 \times n}
& Freq(A) = D + \frac{H}{2} = p
\end{align}
\end{definition}

\begin{definition}[Frequência do alelo A]
\begin{align}
& Freq(a) = \frac{Nº alelos A}{Nº total de alelos}
& Freq(a) = \frac{n3}{n} + \frac{n1}{2 \times n}
& Freq(a) = R + \frac{H}{2} = q
\end{align}
\end{definition}


\begin{example}

Seja uma população com 1000 indivíduos, sendo 400 com genótipo AA, 400 com genótipo Aa, 200 indivíduos com genótipo aa. Calcule a frequência genotípica e frequência genica.

\end{example}

\begin{table}[H]
\centering
\begin{tabular}{l l l l l}
\toprule
 \textbf{Genótipos Possíveis} & \textbf{Frequência (f)} & \textbf{Frequência}  \\
\midrule
 AA &  n1/n & D = 400/1000 = 0,4 \\
 Aa &  n2/n & H = 400/1000 = 0,4 \\
 aa &  n3/n & R = 200/1000 = 0,2 \\
\bottomrule
\end{tabular}
\end{table}


A frquência gênica pode ser calculada obtendo o valor de p e q.

\begin{align}
& Freq(A) = D   + \frac{H}{2} = p
& Freq(A) = 0,4 + \frac{0,4}{2}
& p = 0,6
& q = 1 - p
& q = 1 - 0,6
& q = 0,4
\end{align}


\section{Equilíbrio de Harday-Weinberg (EHW)}

Uma população está em EHW se for suficientemente grande, os acasalamentos forem ao acaso e tiver livre de fatores que alteram a frequência gênica. Assim, as frequÊncias genotípicas e gênicas mantem-se constantes a cada geração.

Os fatores que alteram a frequência gênica podem ser sistemáticos ou dispersivos. 

Os fatores sistemáticos  são aqueles cuja alteração na frequência gênica podem ser conhecidas, tanto em termos de magnitude quanto em direção, como: seleção, migração e mutação. 

Os fatores dispersivos  são aqueles em que é possível conhecer apenas a magnitude da alteração da frequência, mas não a direção em que ela foi alterada, como processo dispersivo é considerado a oscilação genética ou amostragem. 

Quando uma população está em equilíbrio, as frequências genotípicas são dados pelo quadrado da frquência gênica.

\begin{definition}
\begin{align}
& (p+q)^2 = p^2 + 2 \times p \times q + q^2 \\
& Sendo: \\
& p^2 = Freq(AA)  \\
& e  \\
& q^2 = Freq(aa) \\
\end{align}
\end{definition}


\begin{definition}[Demonstração da teoria de EHW]

Para esta demonstração admita:

\begin{enumerate}
\item freq(A) + freq(a) = 1
\item p + q = 1
\item freq(AA) = D
\item freq(Aa) = H
\item freq(aa) = R
\item População de tamanho 'n'
\item Análise de um gene 'A'
\end{enumerate}






\end{definition}